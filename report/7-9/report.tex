\documentclass[a4j,12pt]{jreport}
\usepackage{listings}
\usepackage{plistings}
\usepackage{color}
\usepackage[rm]{roboto}
\usepackage[T1]{fontenc}
\lstset{%
  language={C},
  basicstyle={\small},%
  identifierstyle={\small},%
  commentstyle={\small\itshape\color[rgb]{0,0.5,0}},%
  keywordstyle={\small\bfseries\color[rgb]{0,0,1}},%
  ndkeywordstyle={\small},%
  stringstyle={\small\ttfamily\color[rgb]{1,0,1}},
  frame={tb},
  breaklines=true,
  columns=[l]{fullflexible},%
  numbers=left,%
  xrightmargin=0zw,%
  xleftmargin=3zw,%
  numberstyle={\scriptsize},%
  stepnumber=1,
  numbersep=1zw,%
  lineskip=-0.5ex%
}
\title{基礎情報演習1B第7回~9回 レポート}
\author{AL16030 笠井信宏}
\begin{document}
\maketitle
\chapter{プログラム解説}
\section{RandomWalkプログラム解説}
\subsection{ソースコード記載}
はじめに,randomwalkプログラムで使用したソースコードを記載する.
\lstinputlisting[caption=export.c,label=export]{../../src/7-9/export.c}
\lstinputlisting[caption=imgutil.c,label=imgutil]{../../src/7-9/imgutil.c}
\lstinputlisting[caption=word.c,label=word]{../../src/7-9/word.c}
\lstinputlisting[caption=randomWalk.c,label=randomWalk]{../../src/7-9/randomWalk.c}
なお,ヘッダーファイルに関しては省略した.
\subsection{解説}
基本課題から変更した点はない.randomWalk.c内のinit関数においてコメントアウトされている部分に関しては,課題とは違い出力する色をrandomで決めてもっと多様な画像が出力できるようにした.random関数にて取得した値を足し合わせて割っているのは,全体的に落ち着いた値になるように調整するためである.
\section{Edgeプログラム解説}
\subsection{ソースコード記載}
Edgeプログラムに使用したソースコードを記載する.
\lstinputlisting[caption=edge.c,label=edge]{../../src/7-9/edge.c}
なお,RandomWalkプログラム解説の際に記載したプログラムや,ヘッダファイル,配布ファイルについては省略した.
\subsection{解説}
特筆すべきことはない.tmp\_piexlsにデータを格納することで多少可読性を上げたが,メモリ使用率も同時に上げてしまった.
また,Windows上のmsys2を利用して開発していたところ,一部import.cが動かないと言った問題によりfree関数の利用が適切でない.
\section{thickenプログラム解説}
\subsection{ソースコード記載}
thickenプログラムに使用したソースコードを記載する.
\lstinputlisting[caption=thicken.c,label=thicken]{../../src/7-9/thicken.c}
なお,上記2つのプログラム解説の際に記載したプログラム,ヘッダーファイル,配布ファイルについては省略した.
\subsection{解説}
こちらも特に特筆すべきことはない.edge.cと同じようにtemp\_piexlsを利用してプログラムの可読性を上げた.
\end{document}