\documentclass[a4j,12pt]{jreport}
\usepackage{listings}
\usepackage{plistings}
\usepackage{color}
\lstset{%
  language={C},
  basicstyle={\small},%
  identifierstyle={\small},%
  commentstyle={\small\itshape\color[rgb]{0,0.5,0}},%
  keywordstyle={\small\bfseries\color[rgb]{0,0,1}},%
  ndkeywordstyle={\small},%
  stringstyle={\small\ttfamily\color[rgb]{1,0,1}},
  frame={tb},
  breaklines=true,
  columns=[l]{fullflexible},%
  numbers=left,%
  xrightmargin=0zw,%
  xleftmargin=3zw,%
  numberstyle={\scriptsize},%
  stepnumber=1,
  numbersep=1zw,%
  lineskip=-0.5ex%
}

\title{基礎情報演習1B
第4回~6回レポート}
\author{AL16030 笠井信宏}
\begin{document}
\maketitle
\chapter{プログラム解説}
\section{プログラム記載}
quiz.c,quiz.h,typing.c,typing.h,menu.c,menu.hをそれぞれ載せる.

\begin{lstlisting}[caption=quiz.c,label=quiz.c]
#include <stdlib.h>
#include "quiz.h"
#define N 100

int quizReader(QUIZ *quiz,int *n){
  int i = 0;
  FILE *fp;
  fp = fopen("quiz.txt","r");
  if(fp == NULL){
    printf("ファイルの読み込みに失敗しました\n");
    return -1;
  }
  while(fscanf(fp,"%[^,],%[^,],%[^,],%d",quiz[i].question,quiz[i].ans1,quiz[i].ans2,&quiz[i].correct_ans) != EOF && i < N){
    i++;
  }
  *n = i-1;
  return 0;
}

void input(){
  QUIZ *quizzes = (QUIZ *)malloc(sizeof(QUIZ)*100);
  int num,i,ans,score  = 0;
  if(quizReader(quizzes,&num)){
    printf("エラーが発生したようです\n");
    return;
  }
  for(i=0;i<num;i++){
    printf(quizzes[i].question);
    printf("\n1: %s,2: %s\n",quizzes[i].ans1,quizzes[i].ans2);
    scanf("%d",&ans);
    if(ans == quizzes[i].correct_ans){
      printf("CORRECT!\n");
      score += 5;
    } else {
      printf("WRONG!\n");
    }
    do {
      ans = getchar();
    } while(ans != '\n' && ans != EOF);
    printf("score: %d\n",score);
  }
}
\end{lstlisting}

u.hをそれぞれ載せる.

\begin{lstlisting}[caption=quiz.h,label=quiz.h]
#ifndef QUIZ_H_
#define QUIZ_H_

#include <stdio.h>

struct _quiz {
 char question[50];
 char ans1[50];
 char ans2[50];
 int correct_ans;
};

typedef struct _quiz QUIZ;

int quizReader(QUIZ *quiz, int *n);
void input();

#endif
\end{lstlisting}

\begin{lstlisting}[caption=typing.c,label=typing.c]
#include <stdio.h>
#include <string.h>
#include <stdlib.h>
#include <time.h>
#include "typing.h"
#define MAXLEN 100

int DataReader(DATA *dataArray, int *n){
  FILE *fp;
  int i = 0;
  fp = fopen("typing.txt","r");
  if(fp == NULL) return -1;
  while(fscanf(fp,"%[^,],%d\n",dataArray[i].string,&dataArray[i].score) != EOF || i > MAXLEN){
    i++;
  }
  *n = i - 1;
  return 0;
}

int typing(DATA *dataArray,int n){
  double diff;
  double remain_time = 20.0;
  time_t starttime = time(NULL),endtime;
  int i=1,number,score = 0;

  char ans[50];
  while(1){
    number = rand()%n;
    printf("%s\n",dataArray[number].string);
    // 計測
    starttime = time(NULL);
    scanf("%s",ans);
    endtime = time(NULL);

    fflush(stdin);

    if(strcmp(dataArray[number].string,ans) == 0){
      printf("Great!\n");
      score += dataArray[number].score;
      remain_time += (double)dataArray[number].score;
    } else {
      printf("Bad...\n");
      remain_time -= dataArray[number].score;
    }
    remain_time -= difftime(endtime,starttime);
    if(remain_time > 0){
      printf("%f second left. Your current score is %d\n",remain_time,score);
    } else {
      printf("TIME UP! Game Over...\n");
      break;
    }
  }
  return score;
}
\end{lstlisting}

\begin{lstlisting}[caption=typing.h,label=typing.h]
#ifndef TYPING_H_
#define TYPING_H_

#define MAXLEN 100

struct _data{
  char string[50];
  int score;
};

typedef struct _data DATA;

int DataReader(DATA *dataArray, int *n);
int typing(DATA *dataArray, int n);

#endif  /* TYPING_H_ */
\end{lstlisting}

\begin{lstlisting}[caption=menu.c,label=menu.c]
#include <stdio.h>
#include <stdlib.h>
#include "menu.h"
#define N 100

void clean_stdio(){
  int ch;
  do {
    ch = getchar();
  } while(ch != '\n' && ch != EOF );
}

void menu(){
  DATA *dataArray = (DATA *)malloc(sizeof(DATA)*N);
  int n,success,ret = 1;
  char ch;
  while(ret==1){
    printf("---menu---\nq: Quiz\nt: Typing\ne: Exit\n>");
    ch = getchar();
    switch(ch){
      case 'q':
        input();
        break;
      case 't':
        success = DataReader(dataArray,&n);
        if(!success) typing(dataArray,n);
        break;
      case 'e':
        ret = 0;
        break;
      case '\n':
        break;
      default:
        printf("invalid command\n");
        clean_stdio();
        break;
    }
  }
  free(dataArray);
}
\end{lstlisting}

\begin{lstlisting}[caption=menu.h,label=menu.h]
#ifndef MENU_H_
#define MENU_H_
#include "quiz.h"
#include "typing.h"

void clean_stdio();
void menu();
#endif
\end{lstlisting}

\section*{出力記載}

次に,出力を記載する.
\lstset{%
  basicstyle={\small},%
  ndkeywordstyle={\small},%
  stringstyle={\small\ttfamily\color[rgb]{1,0,1}},
  frame={tb},
  breaklines=true,
  columns=[l]{fullflexible},%
  xrightmargin=0zw,%
  xleftmargin=3zw,%
  numberstyle={\scriptsize},%
  stepnumber=1,
  numbersep=1zw,%
  lineskip=-0.5ex%
}

\begin{lstlisting}[caption="出力"]
$ .\menu.exe
---menu---
q: Quiz
t: Typing
e: Exit
>q
What_is_the_first_month_in_a_year?
1: January,2: December
1
CORRECT!
score: 5

Where_is_the_capital_of_Japan?
1: Kyoto,2: Tokyo
1
WRONG!
score: 5

What_is_the_name_of_this_building?
1: Building_2,2:  Building_4
2
WRONG!
score: 5
---menu---
q: Quiz
t: Typing
e: Exit
>t
AppEngine
AppEngine
Great!
22.000000 second left. Your current score is 4
AppEngine
AppEngine
Great!
22.000000 second left. Your current score is 8
NoSQL
NoSQL
Great!
22.000000 second left. Your current score is 11
NoSQL
NoSQL
Great!
22.000000 second left. Your current score is 14
AppEngine
AppEngine
Great!
23.000000 second left. Your current score is 18
NoSQL
NoSQL
Great!
20.000000 second left. Your current score is 21
MySQL
MYSQL
Bad...
12.000000 second left. Your current score is 21
MySQL
a
Bad...
4.000000 second left. Your current score is 21
NoSQL
a
Bad...
TIME UP! Game Over...
---menu---
q: Quiz
t: Typing
e: Exit
>e
\end{lstlisting}
\section{考察}
今回難点であったのは標準入力のリセットであった.\texttt{getchar}関数を利用する部分において,予期せぬ入力が起きたときに標準入力にバッファが残ってしまい,それがプログラムに予期せぬ動作を起こしていた.それを解決するために利用した\texttt{fflush}関数が,環境事に実装が違うためにエラーが発生していたため,\texttt{clean\_stdio}関数を実装することで解決した.
Windowsでの実行では\texttt{fflush}で問題なかったが,Linuxでは標準入力にテキストが残ってしまっているようだった.
\end{document}